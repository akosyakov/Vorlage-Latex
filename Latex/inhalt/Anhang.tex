%!	Anhang

\clearpage
\appendix
\clearpage

%! Section Befehl wird umgeschrieben, damit keine Überschriften mehr angezeigt werden
%!Kann falls Überschriften gewollt sind entfern werden oder erst später eingefügt
% Beginn 
\renewcommand{\section}[1]{
\par\refstepcounter{section}
\sectionmark{#1}
\addcontentsline{atoc}{section}{\protect\numberline{\thesection}#1}
\lohead{\textnormal{#1}}
} % Ende

%! Hier kann man sich anpassen, wie Abbildungen im Anhang dargestellt werden.
%! Bitte eins der beiden Auskommentieren
%? Möglichkeit 1 ohne Nummerierung und ohne Abbildung davor 
%\renewcommand{\bild}[3][1.0]{\begin{figure}[H]
%	\centering
%	\includegraphics[width=#1\columnwidth]{bilder/#2}
%	\caption*{#3}
%	\label{fig:#3}
%	\end{figure}}

%? Möglichkeit 2 mit Nummerierung und Abbildung, aber nicht im Abbildungsverzeichnis
\renewcommand{\bild}[3][1.0]{\begin{figure}[H]
	\centering
	\includegraphics[width=#1\columnwidth]{bilder/#2}
	\caption[]{#3}
	\label{fig:#3}
	\end{figure}}

%! Anhang 1
\section{Erster toller Anhang}
Hihi hier kommt eigentlich ein Anhangverzeichnis hin :D
\clearpage

%! Anhang 2
\section{Inhalt der CD}
CD mit folgenden Inhalten:
\begin{itemize}
	\item dieses Dokument
	\item Latex Dateien
	\item Youtube-Video als Bonus
\end{itemize}
 \clearpage

%! Eidestattliche Erklärung, hier zwischen Version für einen oder mehrere Autoren umschalten
\input{inhalt/Erklärung_Praxisbeleg}
%\input{inhalt/Erklärung_3Autoren}
